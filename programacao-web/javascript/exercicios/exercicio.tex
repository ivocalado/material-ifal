\documentclass[a4paper,10pt]{article}
\usepackage[latin1]{inputenc}
\usepackage[brazil]{babel}

%opening
\title{Instituto Federal de Educa��o, Ci�ncia e Tecnologia de Alagoas -- IFAL}
\author{Programa��o para Web -- Curso Integrado de Inform�tica - Exerc�cio sobre JavaScript}

\begin{document}

\maketitle

% Forma de resolu��o: \_\_\_\_\_\_\_\_\_\_\_\_\_\_\_\_\_\_\_\_\_\_\_\_\_\_\_\_

% \vspace{2cm}



\textbf{1.} Criar uma p�gina HTML que c�lcule as raizes de uma equa��o de 2� grau com base nos par�mtros a, b e c passados. Os referidos atributos devem ser obtidos a partir de inputs;

\textbf{2.} Criar um formul�rio HTML que apresente campos com as seguintes restri��es:

\textbf{2.1} que impe�a que o usu�rio digite valores que n�o sejam n�meros (dica: ver evento onKeyDown). Toda vez que o usu�rio tentar digitar um valor que n�o seja n�mero ser� exibido numa div uma mensagem informando do erro;

\textbf{2.2} que enquanto o campo contiver menos do que 5 caracteres ir� alterar a formata��o para fundo rosa e borda vermelha

\textbf{2.3} que enquanto o campo contiver apenas uma palavra ir� alterar a formata��o para fundo rosa e borda vermelha

\textbf{3.} Criar uma p�gina HTML que contenha um input que receba um valor num�rico e um bot�o. Quando o usu�rio digitar o valor e clicar no bot�o deve ser exibido numa div o valor da soma de todos os n�meros pares menores que o valor digitado. Deve ser implementado o controle de entrada solicitado na quest�o 2.1.

\textbf{4.} Criar uma p�gina HTML que contenha uma div formatada no estilo de um quadrado com um texto ``testando'' no centro. Quando o usu�rio passar o mouse por cima do quadrado ele deve alterar o texto para ``HTML'' (dica: ver eventos onMouseOver e onMouseOut



\end{document}
