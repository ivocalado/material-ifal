\documentclass{article}

\usepackage{epsfig}
\usepackage{graphicx}
\usepackage{listings}
\usepackage{a4wide}
\usepackage[latin1]{inputenc}
%\usepackage[portuges]{babel}
\usepackage[brazil]{babel}
\usepackage{float}

\title{Programa��o V \\ \small{\textbf{Prof. Alan}} \\ \scriptsize{\textbf{alanpedros@yahoo.com.br}}\\
JavaBean\\}

\begin{document}

\maketitle

\scriptsize

\begin{description}

\item 1 - Crie uma \emph{aplica��o web} no tomcat chamado \emph{JSPBean}, seguindo os
seguintes procedimentos b�sicos:

\begin{enumerate}

\item Criar uma pasta chamada \emph{JSPBean}, dentro da pasta
$C:\setminus Tomcat\setminus webapps$

\item Criar uma pasta chamada \emph{WEB-INF}, dentro da pasta
$C:\setminus Tomcat\setminus webapps\setminus JSPBean$

\item Criar uma pasta chamada \emph{classes}, dentro da pasta
$C:\setminus Tomcat\setminus webapps\setminus JSPBean\setminus
WEB-INF$

\end{enumerate}


\item 2 - Crie um projeto no eclipse, chamado \emph{ProjetoJavaBean}, seguindo os seguintes procedimentos
b�sicos:

\begin{enumerate}

\item Crie uma pasta do tipo \emph{Source Folder}, chamado
\emph{src};

\item Dentro da pasta \emph{src} do item 1, crie um pacote chamado \emph{bean}.

\item Implemente a classe abaixo abaixo:

\begin{verbatim}
package bean;

public class Aluno {
    private String nome;
    private String endereco;
    private String telefone;

    public void setEndereco(String endereco) { this.endereco = endereco; }
    public void setNome(String nome) { this.nome = nome; }
    public void setTelefone(String telefone) { this.telefone = telefone; }

    public String getDados(){
        return "Nome: " + this.nome + " Endere�o " +
                this.endereco + " Telefone: " + this.telefone;
    }

}
\end{verbatim}

\end{enumerate}

\item 3 - Implemente o arquivo ``index.jsp'' abaixo, e salve na pasta $C:\setminus Tomcat\setminus webapps\setminus JSPBean$:

\begin{verbatim}
<HTML>
 <HEAD>
  <TITLE> P�gina Inicial </TITLE>
 </HEAD>

 <BODY>
  <center>
  <FORM METHOD=POST ACTION="TratarDados.jsp">
  <TABLE>
  <TR>
    <TD>Nome:</td><td> <INPUT TYPE="text" NAME="nome"></TD>
  </TR>
  <TR>
    <TD>Endere�o:</td><td> <INPUT TYPE="text" NAME="endereco"></TD>
  </TR>
  <TR>
    <TD>Telefone:</td><td> <INPUT TYPE="text" NAME="telefone"></TD>
  </TR>
  <TR>
    <TD colspan="2" align="center"><INPUT TYPE="submit"></TD>
  </TR>
  </TABLE>
  </FORM>
 </BODY>
</HTML>

\end{verbatim}

\item 4 - Implemente o arquivo ``TratarDados.jsp'' abaixo, e salve na pasta $C:\setminus Tomcat\setminus webapps\setminus JSPBean$:

\begin{verbatim}
 <jsp:useBean id="aluno" class="bean.Aluno"/>
 <jsp:setProperty name="aluno" property="nome" value="<%=request.getParameter("nome")%>"/>
 <jsp:setProperty name="aluno" property="endereco" value="<%=request.getParameter("endereco")%>"/>
 <jsp:setProperty name="aluno" property="telefone" value="<%=request.getParameter("telefone")%>"/>
<HTML>
 <HEAD>
  <TITLE> Teste JavaBean </TITLE>
 </HEAD>
 <BODY>
  <center>
  <TABLE>
  <TR>
    <TD><jsp:getProperty name="aluno" property="dados"/></TD>
  </TR>
  </TABLE>
 </BODY>
</HTML>
\end{verbatim}

\item 5 - Copie tudo que estiver na pasta \emph{lib} do projeto criado no eclipse e
ponha na pasta $C:\setminus Tomcat\setminus webapps\setminus
JSPBean\setminus WEB-INF\setminus classes$

\item 6 - Reinicie o Tomcat ( $C:\setminus Tomcat\setminus bin\setminus classes\setminus tomcat5$);

\item 7 - Chame a p�gina inicial da seguinte forma:
\textbf{http:$//$localhost:8083$/$JSPBean$/$index.jsp}.



\end{description}

\end{document}
