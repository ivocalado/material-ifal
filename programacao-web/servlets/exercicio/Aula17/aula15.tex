\documentclass{article}

\usepackage{epsfig}
\usepackage{graphicx}
\usepackage{listings}
\usepackage{a4wide}
\usepackage[latin1]{inputenc}
%\usepackage[portuges]{babel}
\usepackage[brazil]{babel}
\usepackage{float}

\title{Programa��o V \\ \small{\textbf{Prof. Alan}} \\ \scriptsize{\textbf{alanpedros@yahoo.com.br}}\\
JavaBean 2\\}

\begin{document}

\maketitle

\scriptsize

\begin{description}

\item 1 - Crie uma \emph{aplica��o web} no tomcat chamado \emph{JSPBean2}, seguindo os
seguintes procedimentos b�sicos:

\begin{enumerate}

\item Criar uma pasta chamada \emph{JSPBean2}, dentro da pasta
$C:\setminus Tomcat\setminus webapps$

\item Criar uma pasta chamada \emph{WEB-INF}, dentro da pasta
$C:\setminus Tomcat\setminus webapps\setminus JSPBean2$

\item Criar uma pasta chamada \emph{classes}, dentro da pasta
$C:\setminus Tomcat\setminus webapps\setminus JSPBean2\setminus
WEB-INF$

\end{enumerate}


\item 2 - Crie um projeto no eclipse, chamado \emph{ProjetoJavaBean2}, em seguida, fa�a os seguintes procedimentos:

\begin{enumerate}

\item Crie uma pasta do tipo \emph{Source Folder}, chamado
\emph{src};

\item Dentro da pasta \emph{src} do item 1, crie um pacote chamado \emph{medel}.

\item Dentro da pasta \emph{src} do item 1, crie um pacote chamado \emph{web}.

\item Crie uma pasta \emph{lib}, e ponha a biblioteca
\emph{C:$\setminus$Tomcat$\setminus$common$\setminus$lib$\setminus$servlet-api.jar}
dentro da pasta. Em seguida, dentro do eclipse, clique com o bot�o
direito do mouse em cima do arquivo \emph{servlet-api.jar}, clique
em \emph{Build Path}$\longrightarrow$\emph{Add to Build Path};

\item Implemente sa classes abaixo abaixo:

\begin{verbatim}
package model;

public class Cliente {
    private String nome;
    private String endereco;
    private String telefone;
    private String cidade;
    private String uf;

    public Cliente(String nome, String endereco, String telefone,
            String cidade, String uf) {
        super();
        this.nome = nome;
        this.endereco = endereco;
        this.telefone = telefone;
        this.cidade = cidade;
        this.uf = uf;
    }
    public String getCidade() { return cidade; }
    public String getEndereco() { return endereco; }
    public String getNome() { return nome; }
    public String getTelefone() { return telefone; }
    public String getUf() { return uf; }
}

\end{verbatim}
-------------------------------------------------------------------------------------------------------------------------
\begin{verbatim}
package model;
public class BancoDeDados {
    public static Cliente getCliente(String id){
        if (id.compareTo("2007.0001")==0){
            Cliente cliente = new Cliente("Maria","Av. Fernandes Lima, 711",
                    "82-1234-5678","Macei�","AL");
            return cliente;
        }else if(id.compareTo("2007.0002")==0){
            Cliente cliente = new Cliente("Jo�o","Rua: C�nego Machado, 722",
                    "82-8765-4321","Macei�","AL");
            return cliente;
        }else{
            return null;
        }
    }
}
\end{verbatim}
-------------------------------------------------------------------------------------------------------------------------
\begin{verbatim}
package web;

import java.io.IOException;

import javax.servlet.RequestDispatcher;

import javax.servlet.ServletException;

import javax.servlet.http.HttpServlet;

import javax.servlet.http.HttpServletRequest;

import javax.servlet.http.HttpServletResponse;

import model.BancoDeDados;

import model.Cliente;

public class RecebeRequisicao extends HttpServlet {

    public void doGet(HttpServletRequest request, HttpServletResponse response)
            throws IOException, ServletException {
        response.sendRedirect("acessoNegado.html");
    }

    public void doPost(HttpServletRequest request, HttpServletResponse response)
            throws IOException, ServletException {
        String id = request.getParameter("id");
        Cliente cliente = BancoDeDados.getCliente(id);

        if (cliente != null) {
            request.setAttribute("cliente", cliente);
            RequestDispatcher dispatcher = request
                    .getRequestDispatcher("/WEB-INF/exibirCliente.jsp");
            dispatcher.forward(request, response);
        } else {
            RequestDispatcher dispatcher = request
                    .getRequestDispatcher("/WEB-INF/paginaDeErro.jsp");
            dispatcher.forward(request, response);
        }

    }

}

\end{verbatim}

\end{enumerate}

\item 3 - Implemente o arquivo ``index.jsp'' abaixo, e salve na pasta $C:\setminus Tomcat\setminus webapps\setminus JSPBean2$:

\begin{verbatim}
<HTML>
 <HEAD>
  <TITLE> P�gina Inicial </TITLE>
 </HEAD>
 <BODY>
 <CENTER>
 <FORM METHOD=POST ACTION="./TratarRequisicao"><TABLE>
 <TR>
    <TD>Digite a sua identifica��o: <INPUT TYPE="text" NAME="id"></TD>
 </TR>
 <TR>
    <TD align="center"><INPUT TYPE="submit" VALUE="Fazer busca"></TD>
 </TR>
 </TABLE>
 </FORM>
 </BODY>
</HTML>
\end{verbatim}

\item 4 - Implemente o arquivo ``acessoNegado.jsp'' abaixo, e salve na pasta $C:\setminus Tomcat\setminus webapps\setminus JSPBean2$:

\begin{verbatim}
<HTML>
 <HEAD>
  <TITLE>Acesso Negado</TITLE>
 </HEAD>
 <BODY>
 <CENTER>
 <TR>
    <TD>Acesso Negado</TD>
 </TR>
 </TABLE>
 </BODY>
</HTML>
\end{verbatim}

\item 5 - Implemente o arquivo ``exibirCliente.jsp'' abaixo, e salve na pasta $C:\setminus Tomcat\setminus webapps\setminus JSPBean2\setminus WEB-INF$:
\begin{verbatim}
<jsp:useBean id="cliente" class="model.Cliente" scope="request"/>
<HTML>
 <HEAD>
  <TITLE> Exibindo Cliente </TITLE>
 </HEAD>
 <BODY>
 <center>
 <TABLE>
 <TR>
    <TD>Nome: <jsp:getProperty name="cliente" property="nome" /></TD>
 </TR>
 <TR>
    <TD>Endere�o: <jsp:getProperty name="cliente" property="endereco" /></TD>
 </TR>
 <TR>
    <TD>Telefone: <jsp:getProperty name="cliente" property="telefone" /></TD>
 </TR>
 <TR>
    <TD>Cidade: <jsp:getProperty name="cliente" property="cidade" /></TD>
 </TR>
 <TR>
    <TD>UF: <jsp:getProperty name="cliente" property="uf" /></TD>
 </TR>
 </TABLE>
 </BODY>
</HTML>
\end{verbatim}

\item 6 - Implemente o arquivo ``paginaDeErro.jsp'' abaixo, e salve na pasta $C:\setminus Tomcat\setminus webapps\setminus JSPBean2\setminus WEB-INF$:
\begin{verbatim}
<HTML>
 <HEAD>
  <TITLE> Cliente n�o encontrado </TITLE>
 </HEAD>
 <BODY>
 <CENTER>
 <TR>
    <TD align="center">Cliente n�o encontrado</TD>
 </TR>
 </TABLE>
 </BODY>
</HTML>
\end{verbatim}

\item 7 - Copie tudo que estiver na pasta \emph{lib} do projeto criado no eclipse e
ponha na pasta $C:\setminus Tomcat\setminus webapps\setminus
JSPBean2\setminus WEB-INF\setminus classes$

\item 8 - Reinicie o Tomcat ( $C:\setminus Tomcat\setminus bin\setminus classes\setminus tomcat5$);

\item 9 - Chame a p�gina inicial ( \textbf{http:$//$localhost:8085$/$JSPBean2$/$} ) e digite
a identifica��o ```2007.0001''.

\item 10 - Chame a p�gina inicial ( \textbf{http:$//$localhost:8085$/$JSPBean2$/$} ) e digite
a identifica��o ```2007.0002''.

\item 11 - Chame a p�gina inicial ( \textbf{http:$//$localhost:8085$/$JSPBean2$/$} ) e digite
a identifica��o ```2007.0003''.



\end{description}

\end{document}
