\documentclass{article}

\usepackage{epsfig}
\usepackage{graphicx}
\usepackage{listings}
\usepackage{a4wide}
\usepackage[latin1]{inputenc}
%\usepackage[portuges]{babel}
\usepackage[brazil]{babel}
\usepackage{float}

\title{Programa��o V \\ \small{\textbf{Prof. Alan}} \\ \scriptsize{\textbf{alanpedros@yahoo.com.br}}\\
JSP\\}

\begin{document}

\maketitle

\scriptsize

\begin{description}

\item 1 - Crie uma \emph{aplica��o web} no tomcat chamado \emph{JSPII}, seguindo os
seguintes procedimentos b�sicos:

\begin{enumerate}

\item Criar uma pasta chamada \emph{JSPII}, dentro da pasta
$C:\setminus Tomcat\setminus webapps$

\item Criar uma pasta chamada \emph{WEB-INF}, dentro da pasta
$C:\setminus Tomcat\setminus webapps\setminus JSPII$

\item Criar uma pasta chamada \emph{classes}, dentro da pasta
$C:\setminus Tomcat\setminus webapps\setminus JSPII\setminus
WEB-INF$

\end{enumerate}


\item 2 - Crie um projeto no eclipse, chamado \emph{ProjetoJSPII}, seguindo os seguintes procedimentos
b�sicos:

\begin{enumerate}

\item Crie uma pasta do tipo \emph{Source Folder}, chamado
\emph{src};

\item Crie uma pasta \emph{lib}, e ponha a biblioteca
\emph{C:$\setminus$Tomcat$\setminus$common$\setminus$lib$\setminus$servlet-api.jar}
dentro da pasta. Em seguida, dentro do eclipse, clique com o bot�o
direito do mouse em cima do arquivo \emph{servlet-api.jar}, clique
em \emph{Build Path}$\longrightarrow$\emph{Add to Build Path};

\item Dentro da pasta \emph{src} do item 1, crie um pacote chamado \emph{model}.

\item Implemente as classes abaixo abaixo:

\begin{verbatim}
package model;

public class Aluno {
    private String nome;
    private String endereco;

    public Aluno(String nome, String endereco) {
        this.nome = nome;
        this.endereco = endereco;
    }

    public String getEndereco() {
        return endereco;
    }

    public String getNome() {
        return nome;
    }
}
\end{verbatim}

\begin{verbatim}
package model;

public class Turma {
    private String[] nomes = {"Jos�", "Pedro", "Maria","Josefa"};
    private String[] enderecos = {"Rua: 01 Bairro: A","Rua: 02 Bairro: B",
                                "Rua: 03 Bairro: C","Rua: 04 Bairro: D"};

    public Aluno[] getAlunos(){
        Aluno alunos[] = new Aluno[4];
        for (int i=0;i<4;i++){
            alunos[i] = new Aluno(nomes[i],enderecos[i]);
        }
        return alunos;
    }

    public static void main(String[] args) {
        Turma turma = new Turma();
        Aluno[] alunos = turma.getAlunos();
        for (int i = 0; i < alunos.length; i++) {
            System.out.println(alunos[i].getNome());
        }
    }

}
\end{verbatim}

\end{enumerate}

\item 3 - Implemente o arquivo ``index.jsp'' abaixo, e salve na pasta $C:\setminus Tomcat\setminus webapps\setminus JSPII$:

\begin{verbatim}
<%@ page import="model.Turma, model.Aluno"%>
<HTML>
 <HEAD>
  <TITLE> Gera��o Din�mica de Tabelas </TITLE>
 </HEAD>
 <BODY>
<center>
    <TABLE>

        <%
        Turma turma = new Turma();
        Aluno[] alunos = turma.getAlunos();
        for (int i = 0; i < alunos.length; i++) {
        %>
    <TR>
        <TD><%=alunos[i].getNome()%></TD>
        <TD><%=alunos[i].getEndereco()%></TD>
    </TR>
        <%
        }
        %>
    </TABLE>
 </BODY>
</HTML>
\end{verbatim}

\item 5 - Copie tudo que estiver na pasta \emph{bin} do projeto criado no eclipse e
ponha na pasta $C:\setminus Tomcat\setminus webapps\setminus
JSPII\setminus WEB-INF\setminus classes$

\item 6 - Reinicie o Tomcat ( $C:\setminus Tomcat\setminus bin\setminus classes\setminus tomcat5$);

\item 7 - Chame a p�gina inicial da seguinte forma:
\textbf{http:$//$localhost:8085$/$JSPII$/$index.jsp}.



\end{description}

\end{document}
