\documentclass{article}

\usepackage{graphicx}
\usepackage{listings}
\usepackage{a4wide}
\usepackage[latin1]{inputenc}
%\usepackage[portuges]{babel}
\usepackage[brazil]{babel}
\usepackage{float}

\title{ 
  \begin{center}
  \includegraphics[scale=0.20]{./imagens/ifal-logo.jpg}
  \end{center}
  \normalsize 
  \textbf{Instituto Federal de Alagoas - IFAL \\ Campus Palmeira do �ndios \\  Programa��o Web \\ Prof. Ivo Calado \\}  
}

\begin{document}

\maketitle \scriptsize
\begin{description}

\item 1 - Crie uma p�gina html, \emph{index.html}, conforme abaixo.
Depois salve a p�gina na pasta raiz da aplica��o web PrimeiroServlet
(\emph{Tomcat/webapps/PrimeiroServlet}). Por ultimo, solicite a
seguinte url: \emph{http://localhost:8080/PrimeiroServlet/}

\begin{lstlisting}
<HTML>
 <HEAD>
  <TITLE> Primeiro Formul�rio </TITLE>
 </HEAD>

 <BODY>
  <FORM METHOD=POST ACTION="./ObjetoRequest">
    <TABLE align="center">
    <TR>
    <TD>Nome:</TD>
        <TD><INPUT TYPE="text" NAME="nome"></TD>
    </TR>
    <TR>
    <TD>Telefone:</TD>
        <TD><INPUT TYPE="text" NAME="telefone"></TD>
    </TR>
    <TR>
    <TD>Endereco:</TD>
        <TD><INPUT TYPE="text" NAME="endereco"></TD>
    </TR>
    </TABLE>
    <center>
    <INPUT TYPE="submit" Name="Enviar Dados">
  </FORM>
 </BODY>
</HTML>
\end{lstlisting}

\item 2 - Crie o Servlet abaixo na aplica��o preexistente. Execute, verificando poss�veis erros.

\begin{lstlisting}
package web;

import java.io.IOException; import java.io.PrintWriter;

import javax.servlet.ServletException; import
javax.servlet.http.HttpServlet; import
javax.servlet.http.HttpServletRequest; import
javax.servlet.http.HttpServletResponse;

public class ObjetoRequest extends HttpServlet {
    public void doPost(HttpServletRequest request, HttpServletResponse response)
            throws IOException, ServletException {
        response.setContentType("text/html");
        PrintWriter printWriter = response.getWriter();
        String nome = request.getParameter("nome");
        String endereco = request.getParameter("endereco");
        String telefone = request.getParameter("telefone");
        printWriter.println("Nome: " + nome);
        printWriter.println("Endereco: " + endereco);
        printWriter.println("Telefone: " + telefone);
    }

    public void doGet(HttpServletRequest request, HttpServletResponse response)
            throws IOException, ServletException {
        response.setContentType("text/html");
        PrintWriter printWriter = response.getWriter();
        printWriter.println("Voc� n�o digitou o formul�rio!");
    }

}
\end{lstlisting}
\item 3 - Redistribua a aplica��o para que o novo Servlet possa ser
enxergado a parir de dois caminhos distintos. Ou seja, atualize o arquivo \emph{web.xml}. Execute e tente acessar o servlet a partir das duas URLs.



\end{description}
\end{document}
