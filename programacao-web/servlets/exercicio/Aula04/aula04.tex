\documentclass{article}

\usepackage{epsfig}
\usepackage{graphicx}
\usepackage{listings}
\usepackage{a4wide}
\usepackage[latin1]{inputenc}
%\usepackage[portuges]{babel}
\usepackage[brazil]{babel}
\usepackage{float}

\title{Programa��o V \\ \small{\textbf{Prof. Alan}} \\ \scriptsize{\textbf{alanpedros@yahoo.com.br}}\\
Implementando o primeiro Servlet - Continua��o\\}

\begin{document}

\maketitle \scriptsize

\begin{description}

\item 1 - Crie uma p�gina html, \emph{index.html}, conforme abaixo.
Depois salve a p�gina na pasta raiz da aplica��o web PrimeiroServlet
(\emph{Tomcat/webapps/PrimeiroServlet}). Por ultimo, solicite a
seguinte url: \emph{http://localhost:8080/PrimeiroServlet/}

\begin{lstlisting}
<HTML>
 <HEAD>
  <TITLE> Primeiro Formul�rio </TITLE>
 </HEAD>

 <BODY>
  <FORM METHOD=POST ACTION="./ObjetoRequest">
    <TABLE align="center">
    <TR>
    <TD>Nome:</TD>
        <TD><INPUT TYPE="text" NAME="nome"></TD>
    </TR>
    <TR>
    <TD>Telefone:</TD>
        <TD><INPUT TYPE="text" NAME="telefone"></TD>
    </TR>
    <TR>
    <TD>Endereco:</TD>
        <TD><INPUT TYPE="text" NAME="endereco"></TD>
    </TR>
    </TABLE>
    <center>
    <INPUT TYPE="submit" Name="Enviar Dados">
  </FORM>
 </BODY>
</HTML>
\end{lstlisting}

\item 2 - Crie o Servlet abaixo, e o coloque na pasta \emph{c:$\setminus$Tomcat$\setminus$webapps$\setminus$PrimeiroServlet$\setminus$WEB-INF$\setminus$classes$\setminus$web}

\begin{lstlisting}
package web;

import java.io.IOException; import java.io.PrintWriter;

import javax.servlet.ServletException; import
javax.servlet.http.HttpServlet; import
javax.servlet.http.HttpServletRequest; import
javax.servlet.http.HttpServletResponse;

public class ObjetoRequest extends HttpServlet {
    public void doPost(HttpServletRequest request, HttpServletResponse response)
            throws IOException, ServletException {
        response.setContentType("text/html");
        PrintWriter printWriter = response.getWriter();
        String nome = request.getParameter("nome");
        String endereco = request.getParameter("endereco");
        String telefone = request.getParameter("telefone");
        printWriter.println("Nome: " + nome);
        printWriter.println("Endereco: " + endereco);
        printWriter.println("Telefone: " + telefone);
    }

    public void doGet(HttpServletRequest request, HttpServletResponse response)
            throws IOException, ServletException {
        response.setContentType("text/html");
        PrintWriter printWriter = response.getWriter();
        printWriter.println("Voc� n�o digitou o formul�rio!");
    }

}
\end{lstlisting}
\newpage
\item 3 - Redistribua a aplica��o para que o novo Servlet possa ser
enxergado. Ou seja, atualize o arquivo \emph{web.xml}.
\begin{lstlisting}
<web-app xmlns="http://java.sun.com/xml/ns/j2ee"
    xmlns:xsi="http://www.w3.org/2001/XMLSchema-instance"
    xsi:schemaLocation="http://java.sun.com/xml/ns/j2ee/web-app_2_4.xsd"

>

<servlet>
    <servlet-name>IdentificadorInterno</servlet-name>
    <servlet-class>web.PrimeiroServlet</servlet-class>
</servlet>

<servlet-mapping>
    <servlet-name>IdentificadorInterno</servlet-name>
    <url-pattern>/NomePublico</url-pattern>
</servlet-mapping>

<servlet>
    <servlet-name>IdentificadorInterno2</servlet-name>
    <servlet-class>web.ObjetoRequest</servlet-class>
</servlet>

<servlet-mapping>
    <servlet-name>IdentificadorInterno2</servlet-name>
    <url-pattern>/ObjetoRequest</url-pattern>
</servlet-mapping>

</web-app>
\end{lstlisting}


\end{description}

\end{document}
