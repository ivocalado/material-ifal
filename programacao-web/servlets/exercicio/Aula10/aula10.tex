\documentclass{article}

\usepackage{epsfig}
\usepackage{graphicx}
\usepackage{listings}
\usepackage{a4wide}
\usepackage[latin1]{inputenc}
%\usepackage[portuges]{babel}
\usepackage[brazil]{babel}
\usepackage{float}

\title{Programa��o V \\ \small{\textbf{Prof. Alan}} \\ \scriptsize{\textbf{alanpedros@yahoo.com.br}}\\
Primeiros JSP\\}

\begin{document}

\maketitle

\scriptsize

\begin{description}

\item 1 - Crie uma \emph{aplica��o web} no tomcat chamado \emph{TesteJSP}, seguindo os
seguintes procedimentos b�sicos:

\begin{enumerate}

\item Criar uma pasta chamada \emph{TesteJSP}, dentro da pasta
$C:\setminus Tomcat\setminus webapps$

\item Criar uma pasta chamada \emph{WEB-INF}, dentro da pasta
$C:\setminus Tomcat\setminus webapps\setminus TesteJSP$

\item Criar uma pasta chamada \emph{classes}, dentro da pasta
$C:\setminus Tomcat\setminus webapps\setminus TesteJSP\setminus
WEB-INF$

\item Criar o arquivo \emph{web.xml} dentro da pasta
$C:\setminus Tomcat\setminus webapps\setminus TesteJSP\setminus
WEB-INF$

\end{enumerate}


\item 2 - Crie um projeto no eclipse, chamado
\emph{TesteJSP}, seguindo os seguintes procedimentos b�sicos:

\begin{enumerate}

\item Crie uma pasta do tipo \emph{Source Folder}, chamado
\emph{src};

\item Crie uma pasta \emph{lib}, e ponha a biblioteca
\emph{C:$\setminus$Tomcat$\setminus$common$\setminus$lib$\setminus$servlet-api.jar}
dentro da pasta. Em seguida, dentro do eclipse, clique com o bot�o
direito do mouse em cima do arquivo \emph{servlet-api.jar}, clique
em \emph{Build Path}$\longrightarrow$\emph{Add to Build Path};

\item Dentro da pasta \emph{src} do item 1, crie um pacote chamado \emph{web}.

\item Implemente o Servlet abaixo:

\begin{verbatim}
package web;

import java.io.IOException; import java.io.PrintWriter;

import javax.servlet.ServletException; import
javax.servlet.http.HttpServlet; import
javax.servlet.http.HttpServletRequest; import
javax.servlet.http.HttpServletResponse; import
javax.servlet.http.HttpSession;

public class TratarLoginJSP extends HttpServlet{

    public void doPost(HttpServletRequest request, HttpServletResponse response)
    throws IOException, ServletException {
        response.setContentType("text/html");
        PrintWriter printWriter = response.getWriter();
        String login = request.getParameter("login");
        String senha = request.getParameter("senha");
        if ((login.compareTo("alan") == 0) && (senha.compareTo("1234") == 0)) {
            HttpSession httpSession = request.getSession();
            httpSession.setAttribute("login", "alan");
            httpSession.setAttribute("senha", "1234");
            response.sendRedirect("loginCorreto.jsp");
        }else{
            HttpSession httpSession = request.getSession();
            httpSession.setAttribute("mensagem","Login ou senha inv�lida!");
            response.sendRedirect("index.jsp");
        }
    }

}

\end{verbatim}

\item Implemente o arquivo ``index.jsp'' abaixo, e salve na pasta $C:\setminus Tomcat\setminus webapps\setminus TesteJSP$:

\begin{verbatim}
<%
String mensagem = (String)session.getAttribute("mensagem"); if
(mensagem==null){
    mensagem="";
}
%>

<HTML>
 <HEAD>
  <TITLE> Pagina Inicial </TITLE>
 </HEAD>
 <BODY>

<center>
 <FORM METHOD=POST ACTION="./TratarLoginJSP">
    <TABLE>
    <TR>
        <TD>Login:</TD>
        <TD><INPUT TYPE="text" NAME="login"></TD>
    </TR>
    <TR>
        <TD>Senha:</TD>
        <TD><INPUT TYPE="password" NAME="senha"></TD>
    </TR>
    <TR>
        <TD><INPUT TYPE="submit"></TD>
        <TD></TD>
    </TR>
    <TR>
        <TD><%=mensagem%></TD>
        <TD></TD>
    </TR>
    </TABLE>
 </FORM>
 </BODY>
</HTML>
\end{verbatim}

\item Implemente o arquivo ``loginCorreto.jsp'' abaixo, e salve na pasta $C:\setminus Tomcat\setminus webapps\setminus TesteJSP$:

\begin{verbatim}
<%
String login = (String)session.getAttribute("login"); String senha =
(String)session.getAttribute("senha");
%>

<HTML>
 <HEAD>
  <TITLE> Pagina Inicial </TITLE>
 </HEAD>
 <BODY>

<center>
    <TABLE>
    <TR>
        <TD>Login:</TD>
        <TD><%=login%></TD>
    </TR>
    <TR>
        <TD>Senha:</TD>
        <TD><%=senha%></TD>
    </TR>
    <TR>
        <TD></TD>
        <TD><%=application.getInitParameter("parametro1")%></TD>
    </TR>
    </TABLE>
 </BODY>
</HTML>

\end{verbatim}

\end{enumerate}

\item 3 - Crie o arquivo \emph{web.xml} dentro da pasta
$C:\setminus Tomcat\setminus webapps \setminus TesteJSP \setminus
WEB-INF$

\begin{verbatim}
<web-app xmlns="http://java.sun.com/xml/ns/j2ee"
    xmlns:xsi="http://www.w3.org/2001/XMLSchema-instance"
    xsi:schemaLocation="http://java.sun.com/xml/ns/j2ee/web-app_2_4.xsd">

<context-param>
    <param-name>parametro1</param-name>
    <param-value>Seja bem vindo!</param-value>
</context-param>

<servlet>
    <servlet-name>Servlet01</servlet-name>
    <servlet-class>web.TratarLoginJSP</servlet-class>
</servlet>

<servlet-mapping>
    <servlet-name>Servlet01</servlet-name>
    <url-pattern>/TratarLoginJSP</url-pattern>
</servlet-mapping>


</web-app>
\end{verbatim}

\item 4 - Chame a p�gina inicial da seguinte forma:
\textbf{http:$//$localhost:8085$/$TesteJSP$/$index.jsp}. Na primeira
tentativa, digite uma senha e login qualquer. Na segunda, Login
``alan'' e senha ``1234''.



\end{description}

\end{document}
