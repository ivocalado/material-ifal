\documentclass{article}

\usepackage{epsfig}
\usepackage{graphicx}
\usepackage{listings}
\usepackage{a4wide}
\usepackage[latin1]{inputenc}
%\usepackage[portuges]{babel}
\usepackage[brazil]{babel}
\usepackage{float}

\title{Programa��o Web\\ \small{\textbf{Prof. Alan}} \\ \scriptsize{\textbf{alanpedros@yahoo.com.br}}\\
Combo Box Din�mico\\}

\begin{document}

\maketitle

\scriptsize

\begin{description}

\item 1 - Crie uma \emph{aplica��o web} no tomcat chamado \emph{ComboBoxDinamico}, seguindo os
seguintes procedimentos b�sicos:

\begin{enumerate}

\item Criar uma pasta chamada \emph{ComboBoxDinamico}, dentro da pasta
$C:\setminus Tomcat\setminus webapps$

\item Criar uma pasta chamada \emph{WEB-INF}, dentro da pasta
$C:\setminus Tomcat\setminus webapps\setminus TesteJSP$

\item Criar uma pasta chamada \emph{classes}, dentro da pasta
$C:\setminus Tomcat\setminus webapps\setminus TesteJSP\setminus
WEB-INF$

\item Criar o arquivo \emph{web.xml} dentro da pasta
$C:\setminus Tomcat\setminus webapps\setminus TesteJSP\setminus
WEB-INF$

\end{enumerate}


\item 2 - Crie um projeto no eclipse, chamado
\emph{ComboBoxDinamico}, seguindo os seguintes procedimentos
b�sicos:

\begin{enumerate}

\item Crie uma pasta do tipo \emph{Source Folder}, chamado
\emph{src};

\item Crie uma pasta \emph{lib}, e ponha a biblioteca
\emph{C:$\setminus$Tomcat$\setminus$common$\setminus$lib$\setminus$servlet-api.jar}
dentro da pasta. Em seguida, dentro do eclipse, clique com o bot�o
direito do mouse em cima do arquivo \emph{servlet-api.jar}, clique
em \emph{Build Path}$\longrightarrow$\emph{Add to Build Path};

\item Dentro da pasta \emph{src} do item 1, crie um pacote chamado
\emph{model};

\item Implemente a classe \emph{Turma} e \emph{Turmas} abaixo:

\begin{verbatim}
package model;

public class Turma {
    private int codigo;
    private String descricao;

    public Turma(int codigo, String descricao) {
        this.codigo = codigo;
        this.descricao = descricao;
    }

    public int getCodigo() {
        return codigo;
    }

    public String getDescricao() {
        return descricao;
    }

    public void setCodigo(int codigo) {
        this.codigo = codigo;
    }

    public void setDescricao(String descricao) {
        this.descricao = descricao;
    }
}
\end{verbatim}

\begin{verbatim}
package model;

import java.util.Iterator;

import java.util.Vector;

public class Turmas {
    private Vector turmas;

    public Turmas(){
        Turma turma01 = new Turma(01,"Inform�tica");
        Turma turma02 = new Turma(02,"Edifica��es");
        this.turmas = new Vector();
        this.turmas.add(turma01);
        this.turmas.add(turma02);
    }

    public Turma[] getTurmas(){
        Iterator listaDeTurmas = turmas.iterator();
        Turma[] turmas = new Turma[this.turmas.size()];
        int i=0;
        while (listaDeTurmas.hasNext()){
            turmas[i++]=(Turma)listaDeTurmas.next();
        }
        return turmas;
    }


}
\end{verbatim}

\item Coloque todo o conte�do da pasta \emph{bin} do seu projeto do
eclipse, e ponha na pasta classes da sua nova aplica��o web
(ComboBoxDinamico);

\item Implemente o arquivo ``comboboxdinamico.jsp'' abaixo, e salve na pasta $C:\setminus Tomcat\setminus webapps\setminus
ComboBoxDinamico$:;

\begin{verbatim}
<%@page import="model.Turmas, model.Turma" %>
<%@page contentType="text/html" pageEncoding="UTF-8"%>
<!DOCTYPE HTML PUBLIC "-//W3C//DTD HTML 4.01 Transitional//EN"
   "http://www.w3.org/TR/html4/loose.dtd">

<html>
    <head>
        <meta http-equiv="Content-Type" content="text/html; charset=UTF-8">
        <title>Combo Box Din�mico</title>
    </head>
    <body>
        <form method="post" action="./comboxdinamico_.jsp">
        <table align="center">
            <tr><td>
            <select name="abc">
                <%
                    Turmas turmas= new Turmas();
                    Turma[] turma = turmas.getTurmas();
                    for (int i=0;i<turma.length;i++) {

                %>
                <option value="<%=turma[i].getCodigo()%>"><%=turma[i].getDescricao()%></option>
                <%
                    }
                %>
            </select>
            </tr><td>
            <tr><td><input type="submit"></td></tr>

        </table>


        </form>
    </body>
</html>
\end{verbatim}

\item Implemente o arquivo ``comboboxdinamico\_.jsp'' abaixo, e salve na pasta $C:\setminus Tomcat\setminus webapps\setminus ComboBoxDinamico$:

\begin{verbatim}
<%@page contentType="text/html" pageEncoding="UTF-8"%>
<!DOCTYPE HTML PUBLIC "-//W3C//DTD HTML 4.01 Transitional//EN"
   "http://www.w3.org/TR/html4/loose.dtd">

<html>
    <head>
        <meta http-equiv="Content-Type" content="text/html; charset=UTF-8">
        <title>Op��es Escolhidas do CheckBox Din�mico</title>
    </head>
    <body>
    <table align="center">
    <tr><td>Op��es Escolhidas</td></tr>
    <%
    String selectEscolhido = request.getParameter("abc");
    if (selectEscolhido!=null) {
    %>
    <tr><td><%= selectEscolhido   %></td></tr>
    <%
    }else {
    %>
    <tr><td>N�o foi escolhida nenhuma op��o!</td></tr>
    <%}%>
    </table>

   </body>
</html>
\end{verbatim}

\end{enumerate}

\item 4 - Chame a p�gina inicial da seguinte forma:
\textbf{http:$//$localhost:8080$/$ComboBoxDinamico$/$index.jsp} e
fa�a os devidos testes.



\end{description}

\end{document}
