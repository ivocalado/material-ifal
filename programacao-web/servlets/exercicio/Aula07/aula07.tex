\documentclass{article}

\usepackage{epsfig}
\usepackage{graphicx}
\usepackage{listings}
\usepackage{a4wide}
\usepackage[latin1]{inputenc}
%\usepackage[portuges]{babel}
\usepackage[brazil]{babel}
\usepackage{float}

\title{Programa��o V \\ \small{\textbf{Prof. Alan}} \\ \scriptsize{\textbf{alanpedros@yahoo.com.br}}\\
Gerenciamento de Se��o sem Cookies\\}

\begin{document}

\maketitle

\scriptsize

\begin{description}

\item 1 - No menu \emph{ferramentas} do Internet Explorer, clique em \emph{Op��es
da internet} e em seguida abra a guia \emph{privacidade}. Nessa
guia, configure o navegador para \emph{bloquear todos os cookies}.

\item 2 - Crie uma \emph{aplica��o web} no tomcat chamado \emph{TesteSession2}, seguindo os
seguintes procedimentos b�sicos:

\begin{enumerate}

\item Criar uma pasta chamada \emph{TesteSession2}, dentro da pasta
$C:\setminus Tomcat\setminus webapps$

\item Criar uma pasta chamada \emph{WEB-INF}, dentro da pasta
$C:\setminus Tomcat\setminus webapps \setminus TesteSession2$

\item Criar uma pasta chamada \emph{classes}, dentro da pasta
$C:\setminus Tomcat\setminus webapps \setminus TesteSession2
\setminus WEB-INF$

\item Criar o arquivo \emph{web.xml} dentro da pasta
$C:\setminus Tomcat\setminus webapps \setminus TesteSession2
\setminus WEB-INF$

\end{enumerate}


\item 3 - Crie o arquivo \emph{index.html} conforme abaixo, e ponha
na pasta:
C:$\setminus$Tomcat$\setminus$webapps$\setminus$TesteSession2

\begin{verbatim}
<HTML>
 <HEAD>
  <TITLE> P�gina Inicial </TITLE>
 </HEAD>
 <BODY>
    <FORM METHOD="POST" ACTION="./FazerLogin" ALIGN="CENTER">
        <TABLE align="center">
            <TR>
                <TD>
                    Login:<INPUT TYPE="text" NAME="login">
                </TD>
            </TR>
            <TR>
                <TD>
                    Senha:<INPUT TYPE="password" NAME="senha">
                </TD>
            </TR>
            <TR>
                <TD><INPUT TYPE="submit" Value="Fazer Login"></TD>
            </TR>
            </TABLE>
    </FORM>
 </BODY>
</HTML>
\end{verbatim}

\item 4 - Crie um projeto no eclipse, chamado
\emph{ProjetoTesteSession2}, seguindo os seguintes procedimentos
b�sicos:

\begin{enumerate}

\item Crie uma pasta do tipo \emph{Source Folder}, chamado
\emph{src};

\item Crie uma pasta \emph{lib}, e ponha a biblioteca
\emph{C:$\setminus$Tomcat$\setminus$common$\setminus$lib$\setminus$servlet-api.jar}
dentro da pasta. Em seguida, dentro do eclipse, clique com o bot�o
direito do mouse em cima do arquivo \emph{servlet-api.jar}, clique
em \emph{Build Path}$\longrightarrow$\emph{Add to Build Path};

\item Dentro da pasta \emph{src} do item 1, crie um pacote chamado \emph{web}.

\item Implemente os tr�s servlet's abaixo:

\begin{verbatim}
public class LoginCorreto extends HttpServlet {
    public void doGet(HttpServletRequest request, HttpServletResponse response)
            throws IOException, ServletException {
        response.setContentType("text/html");
        PrintWriter printWriter = response.getWriter();
        HttpSession httpSession = request.getSession();

        try {
            if (((String) httpSession.getAttribute("login")).compareTo("alan") == 0
                    && ((String) httpSession.getAttribute("senha"))
                            .compareTo("1234") == 0) {
                String saida = "<HTML>";
                saida += "<HEAD>";
                saida += "<TITLE> Seja bem vindo ";
                saida += " </TITLE>";
                saida += "</HEAD>";
                saida += "<BODY>";
                saida += "<CENTER><H1>Seja bem vindo</H1></CENTER><a href=\"";
                saida += response.encodeURL(request.getContextPath()
                        + "/acessonegado.html");
                saida += "\">Link</a></BODY>";
                saida += "</HTML>";
                printWriter.print(saida);
            }
        } catch (Exception e) {
            String saida = "<HTML>";
            saida += "<HEAD>";
            saida += "<TITLE> Acesso Negado  </TITLE>";
            saida += "</HEAD>";
            saida += "<BODY>";
            saida += "<CENTER><H1>ERRO</H1></CENTER><a href=\"";
            saida += response.encodeURL(request.getContextPath()
                    + "/acessonegado.html");
            saida += "\">Link</a></BODY>";

            saida += "</HTML>";
            printWriter.print(saida);
        }
    }

    public void doPost(HttpServletRequest request, HttpServletResponse response)
            throws IOException, ServletException {
        response.setContentType("text/html");
        PrintWriter printWriter = response.getWriter();
        String saida = "<HTML>";
        saida += "<HEAD>";
        saida += "<TITLE> Acesso Negado </TITLE>";
        saida += "</HEAD>";
        saida += "<BODY>";
        saida += "<CENTER><H1>Acesso Negado</H1></CENTER>";
        saida += "</BODY>";
        saida += "</HTML>";
        printWriter.print(saida);
    }
}

\end{verbatim}

\begin{verbatim}
public class LoginIncorreto extends HttpServlet{

    public void doGet(HttpServletRequest request, HttpServletResponse response)
            throws IOException, ServletException {
        response.setContentType("text/html");
        PrintWriter printWriter = response.getWriter();
        String saida = "<HTML>";
        saida += "<HEAD>";
        saida += "<TITLE> Login Incorreto </TITLE>";
        saida += "</HEAD>";
        saida += "<BODY>";
        saida += "<CENTER><H1>Login Incorreto!</H1></CENTER>";
        saida += "</BODY>";
        saida += "</HTML>";
        printWriter.print(saida);

    }

    public void doPost(HttpServletRequest request, HttpServletResponse response)
            throws IOException, ServletException {
        response.setContentType("text/html");
        PrintWriter printWriter = response.getWriter();
        String saida = "<HTML>";
        saida += "<HEAD>";
        saida += "<TITLE> Acesso Negado </TITLE>";
        saida += "</HEAD>";
        saida += "<BODY>";
        saida += "<CENTER><H1>Acesso Negado</H1></CENTER>";
        saida += "</BODY>";
        saida += "</HTML>";
        printWriter.print(saida);
    }

}

\end{verbatim}

\begin{verbatim}
public class ServletInserirSessao extends HttpServlet {
    public void doPost(HttpServletRequest request, HttpServletResponse response)
            throws IOException, ServletException {
        response.setContentType("text/html");
        PrintWriter printWriter = response.getWriter();
        String login = request.getParameter("login");
        String senha = request.getParameter("senha");
        if ((login.compareTo("alan") == 0) && (senha.compareTo("1234") == 0)) {
            HttpSession httpSession = request.getSession();
            httpSession.setAttribute("login", "alan");
            httpSession.setAttribute("senha", "1234");
            httpSession.setAttribute("nome", "Alan Pedro da Silva");
            response.sendRedirect(response.encodeURL(request.getContextPath()
                    + "/paginainicial.html"));
        } else {
            response.sendRedirect("acessonegado.html");
        }
    }

    public void doGet(HttpServletRequest request, HttpServletResponse response)
            throws IOException, ServletException {
        response.setContentType("text/html");
        PrintWriter printWriter = response.getWriter();
        String saida = "<HTML>";
        saida += "<HEAD>";
        saida += "<TITLE> Acesso Negado </TITLE>";
        saida += "</HEAD>";
        saida += "<BODY>";
        saida += "<CENTER><H1>Acesso Negado</H1></CENTER>";
        saida += "</BODY>";
        saida += "</HTML>";
        printWriter.print(saida);
    }
}

\end{verbatim}


\end{enumerate}

\item 5 - Crie o arquivo \emph{web.xml} dentro da pasta
$C:\setminus Tomcat\setminus webapps \setminus TesteSession2
\setminus WEB-INF$

\begin{verbatim}
<web-app xmlns="http://java.sun.com/xml/ns/j2ee"
    xmlns:xsi="http://www.w3.org/2001/XMLSchema-instance"
    xsi:schemaLocation="http://java.sun.com/xml/ns/j2ee/web-app_2_4.xsd"

>

<servlet>
    <servlet-name>InserirSession</servlet-name>
    <servlet-class>web.ServletInserirSessao</servlet-class>
</servlet>
<servlet>
    <servlet-name>PaginaCorreta</servlet-name>
    <servlet-class>web.LoginCorreto</servlet-class>
</servlet>
<servlet>
    <servlet-name>PaginaLoginErrado</servlet-name>
    <servlet-class>web.LoginIncorreto</servlet-class>
</servlet>
<servlet-mapping>
    <servlet-name>InserirSession</servlet-name>
    <url-pattern>/FazerLogin</url-pattern>
</servlet-mapping>
<servlet-mapping>
    <servlet-name>PaginaCorreta</servlet-name>
    <url-pattern>/paginainicial.html</url-pattern>
</servlet-mapping>
<servlet-mapping>
    <servlet-name>PaginaLoginErrado</servlet-name>
    <url-pattern>/acessonegado.html</url-pattern>
</servlet-mapping>


</web-app>
\end{verbatim}

\item 6 - Chame a p�gina inicial da seguinte forma:
\textbf{http:$\setminus\setminus$xxx.xxx.xxx.xxxx:8085$\setminus$TesteSession2$\setminus$}
e n�o
\textbf{http:$\setminus\setminus$localhost:8085$\setminus$TesteSession2$\setminus$}.
Pois, � necess�rio simular uma requisi��o externa. Sendo
$xxx.xxx.xxx.xxxx:8085$ seja o IP da sua m�quina.


\end{description}

\end{document}
